% Options for packages loaded elsewhere
\PassOptionsToPackage{unicode}{hyperref}
\PassOptionsToPackage{hyphens}{url}
\PassOptionsToPackage{dvipsnames,svgnames*,x11names*}{xcolor}
%
\documentclass[
]{article}
\usepackage{lmodern}
\usepackage{amssymb,amsmath}
\usepackage{ifxetex,ifluatex}
\ifnum 0\ifxetex 1\fi\ifluatex 1\fi=0 % if pdftex
  \usepackage[T1]{fontenc}
  \usepackage[utf8]{inputenc}
  \usepackage{textcomp} % provide euro and other symbols
\else % if luatex or xetex
  \usepackage{unicode-math}
  \defaultfontfeatures{Scale=MatchLowercase}
  \defaultfontfeatures[\rmfamily]{Ligatures=TeX,Scale=1}
  \setmainfont[]{Helvetica}
  \setmonofont[Scale=0.8]{Source Code Pro}
\fi
% Use upquote if available, for straight quotes in verbatim environments
\IfFileExists{upquote.sty}{\usepackage{upquote}}{}
\IfFileExists{microtype.sty}{% use microtype if available
  \usepackage[]{microtype}
  \UseMicrotypeSet[protrusion]{basicmath} % disable protrusion for tt fonts
}{}
\makeatletter
\@ifundefined{KOMAClassName}{% if non-KOMA class
  \IfFileExists{parskip.sty}{%
    \usepackage{parskip}
  }{% else
    \setlength{\parindent}{0pt}
    \setlength{\parskip}{6pt plus 2pt minus 1pt}}
}{% if KOMA class
  \KOMAoptions{parskip=half}}
\makeatother
\usepackage{xcolor}
\IfFileExists{xurl.sty}{\usepackage{xurl}}{} % add URL line breaks if available
\IfFileExists{bookmark.sty}{\usepackage{bookmark}}{\usepackage{hyperref}}
\hypersetup{
  colorlinks=true,
  linkcolor=white,
  filecolor=Maroon,
  citecolor=Blue,
  urlcolor=cyan,
  pdfcreator={LaTeX via pandoc}}
\urlstyle{same} % disable monospaced font for URLs
\setlength{\emergencystretch}{3em} % prevent overfull lines
\providecommand{\tightlist}{%
  \setlength{\itemsep}{0pt}\setlength{\parskip}{0pt}}
\setcounter{secnumdepth}{-\maxdimen} % remove section numbering
\usepackage{unicode-math}

\date{}

\begin{document}

\hypertarget{ordinary-differential-equations}{%
\section{14. Ordinary differential
equations}\label{ordinary-differential-equations}}

\hypertarget{last-time}{%
\subsection{Last time}\label{last-time}}

\begin{itemize}
\item
  Numerical integration (quadrature)
\item
  Interpolate then integrate
\item
  Newton--Cotes methods (equally-spaced points)
\item
  Error analysis
\end{itemize}

\hypertarget{goals-for-today}{%
\subsection{Goals for today}\label{goals-for-today}}

\begin{itemize}
\item
  Ordinary differential equations: review
\item
  Euler method
\end{itemize}

\hypertarget{ordinary-differential-equations-1}{%
\subsection{Ordinary differential
equations}\label{ordinary-differential-equations-1}}

\begin{itemize}
\item
  \textbf{Ordinary differential equation} (ODEs):

  \begin{quote}
  implicitly defines a function \(x(t)\) by relating \(x\) and its
  derivatives
  \end{quote}

  . . .
\item
  e.g.

  \[\dot{x} = -\lambda x\]
\item
  Models radioactive decay: \(x(t)\) is proportion of radioactive nuclei
  at time \(t\)

  . . .
\item
  Need to start somewhere: Initial condition \(x(t=t_0) = x_0\)
\end{itemize}

\hypertarget{solution-to-an-ode}{%
\subsection{Solution to an ODE}\label{solution-to-an-ode}}

\begin{itemize}
\item
  More general: \(\dot{x} = f(x)\)

  . . .
\item
  This means

  \[\dot{x}(t) = f(x(t)) \quad \forall t \in [t_0, t_\text{final}]\]

  . . .
\item
  Tells us how fast solution changes if currently at given value

  . . .
\item
  Together with the initial condition, this \textbf{implicitly}
  determines the value of \(x(t)\) at all times \(t\)

  . . .

  ~\\
\item
  Solution is a \textbf{function} \(x(t)\) for
  \(t \in [t_0, t_\text{final}]\)
\end{itemize}

\hypertarget{existence-and-uniqueness}{%
\subsection{Existence and uniqueness}\label{existence-and-uniqueness}}

\begin{itemize}
\item
  Does ODE \emph{really} specify a solution \(x(t)\)?

  . . .

  ~\\
\item
  (Usually) \textbf{yes}!:
  \href{https://en.wikipedia.org/wiki/Picard\%E2\%80\%93Lindel\%C3\%B6f_theorem}{Existence
  and uniqueness theorem}

  . . .

  ~\\
\item
  Sufficient condition is \(f \in C^1\) (continuous first derivative)
\end{itemize}

\hypertarget{meaning-of-an-ode}{%
\subsection{Meaning of an ODE}\label{meaning-of-an-ode}}

\begin{itemize}
\item
  Start at initial condition (known)
\item
  ODE tells us derivative of solution at that point

  . . .
\item
  So we have tangent vector to the solution at \(t=t_0\)
\item
  i.e.~we know \emph{in which direction} we should move

  . . .
\item
  As soon as we move a bit, must change to new direction!
\end{itemize}

\hypertarget{euler-method}{%
\subsection{Euler method}\label{euler-method}}

\begin{itemize}
\item
  This suggests a \textbf{numerical method}
\item
  We literally follow the above prescription
\item
  Need method to approximate the true, unknown solution
\item
  One possible way (but not the only one, by any means) is to creep
  forward in small time steps
\end{itemize}

\hypertarget{time-stepping}{%
\subsection{Time stepping}\label{time-stepping}}

\begin{itemize}
\item
  Take equal time steps of length \(h\) (for now)
\item
  So \(t_n := t_n + n \, h\)
\item
  Want to approximate true solution \(x(t_n)\)
\item
  Call approximation \(x_n\)
\item
  Calculate sequence of approximations \(x_0\), \(x_1\), \(\ldots\),
  \(x_N\)
\end{itemize}

\hypertarget{euler-method-1}{%
\subsection{Euler method}\label{euler-method-1}}

\begin{itemize}
\item
  Simplest idea: Suppose derivative constant over time step
\item
  Equivalent: approximate \(f(x(s))\) by constant function
\item
  So \(x_1 - x_0 = \int_{t_0}^{t_1} f(x_0) \, ds = h f(x_0)\)
\item
  Compare rectangular rule for integration
\item
  Repeating this for each step gives the \textbf{Euler method}:

  \[x_{n+1} = x_n + h \, f(x_n)\]
\end{itemize}

\hypertarget{convergence-of-euler-method}{%
\subsection{Convergence of Euler
method}\label{convergence-of-euler-method}}

\begin{itemize}
\item
  A proposed numerical method like this is worthwhile only if it is
  \textbf{convergent}:
\item
  Call \(x_{n, h}\) the solution \(x_n\) with time-step \(h\)
\item
  As \(h \to 0\), the solution produced by the Euler method, i.e.~the
  collection of values \((x_{0, h}, x_{1, h}, \ldots, x_{N(h), h})\),
  should converge to the true solution
  \((x(t_0), x(t_1), \ldots, x(t_N))\)
\item
  I.e. maximum distance should \(\to 0\) as \(h \to 0\)
\item
  This can be proved correct: See e.g.~Iserles, \emph{A First Course in
  the Numerical Analysis of Differential Equations}
\end{itemize}

\hypertarget{rate-of-convergence}{%
\subsection{Rate of convergence}\label{rate-of-convergence}}

\begin{itemize}
\item
  Want \textbf{rate of convergence} of error as function of \(h\)
\item
  Look at single step and suppose start at exact value:
\end{itemize}

\begin{align*}
    x(t_{n+1}) &= x(t_n) + h \, \dot{x}(t_n) + \frac{1}{2}   h^2 \ddot{x}(\xi) \\
    &= x(t_n) + h f(x(t_n)) + h^2 \ddot{x}(\xi)
\end{align*}

\begin{itemize}
\item
  So local error is \(\mathcal{O}(h^2)\) at each step -- \textbf{order
  1}
\item
  There are \(N \sim \frac{1}{h}\) steps so expect global error to be
  \(\mathcal{O}(h)\)
\end{itemize}

\hypertarget{inhomogeneous-odes}{%
\subsection{Inhomogeneous ODEs}\label{inhomogeneous-odes}}

\begin{itemize}
\item
  \(f\) can depend on time too:

  \[\dot{x}(t) = f(t, x(t))\]
\item
  E.g. if there is periodic external forcing

  . . .

  ~\\
\item
  Then Euler method becomes

  \[x_{n+1} = x_n + h_n \, f(t_n, x_n)\]

  in general case with different step sizes \(h_n\)
\end{itemize}

\hypertarget{alternative-trapezium-rule}{%
\subsection{Alternative: Trapezium
rule}\label{alternative-trapezium-rule}}

\begin{itemize}
\item
  Can approximate integral using any quadrature method

  . . .
\item
  E.g. trapezium rule:

  \[x_{n+1} = x_n + \textstyle \frac{h}{2} \left[ f(x_n) + f(x_{n+1}) \right]\]

  . . .

  ~\\
\item
  How can we find \(x_{n+1}\)? Now \textbf{implicit} method

  . . .
\item
  Must solve nonlinear equation at each step, e.g.~using Newton method

  . . .
\item
  More expensive but necessary under certain (common) circumstances:
  \textbf{stiff equations} (see later)

  . . .
\item
  Local error is \(\mathcal{O}(h^3)\) and global is \(\mathcal{O}(h^2)\)
\end{itemize}

\hypertarget{systems-of-equations}{%
\subsection{Systems of equations}\label{systems-of-equations}}

\begin{itemize}
\item
  Usually there will be \(>1\) dependent variable, e.g.

  \begin{align*}
    \dot{x} &= f(x, y) \\
    \dot{y} &= g(x, y)
    \end{align*}
\item
  \(x\) and \(y\) are \textbf{coupled} together
\item
  So \textbf{cannot} solve equations independently
\item
  Rewrite in vector form:

  \[\dot{\mathbf{x}} = \mathbf{f}(\mathbf{x})\]

  \[\dot{\mathbf{x}}(t) = \mathbf{f}(t, \mathbf{x}(t))\]
\item
  \(\mathbf{f}\) is now a \textbf{vector field}
\end{itemize}

\hypertarget{solving-systems-of-equations}{%
\subsection{Solving systems of
equations}\label{solving-systems-of-equations}}

\begin{itemize}
\item
  \(\mathbf{x}(t) = (x_1(t), x_2(t), \ldots, x_d(t))\) if \(d\)
  variables
\item
  Taylor expand:

  \[x_i(t_k + h) = x_i(t_k) + h \, \dot{x_i}(t_k) + \mathcal{O}(h^2)\]
\item
  Get Euler method

  \[\mathbf{x}_{k+1} = \mathbf{x}_k + h \, \mathbf{f}(\mathbf{x}_k)\]
\item
  \emph{Same} method but now with vectors
\item
  \emph{Same} code
\end{itemize}

\hypertarget{higher-derivatives}{%
\subsection{Higher derivatives}\label{higher-derivatives}}

\begin{itemize}
\item
  How should we deal with higher-order equations (higher derivatives)?
\item
  E.g. damped harmonic oscillator

  \[\ddot{x} + b \dot{x} + \omega^2 x = 0\]

  . . .
\item
  There are some special methods for second-order
\item
  But common to reduce to previous case:
\item
  Introduce new variable \(v := \dot{x}\) so
\end{itemize}

\textbackslash begin\{align*\} \dot{x} \&= v \textbackslash{} \dot{v}
\&= -b v + \omega\^{}2 x \emd{align*}`

\begin{verbatim}
since $\dot{v} = \ddot{x}$
\end{verbatim}

\hypertarget{summary}{%
\subsection{Summary}\label{summary}}

\begin{itemize}
\item
  Reviewed ordinary differential equations (ODEs)
\item
  Solution is a function
\item
  Approximate solution using time stepping: Euler method
\end{itemize}

\end{document}
