% Options for packages loaded elsewhere
\PassOptionsToPackage{unicode}{hyperref}
\PassOptionsToPackage{hyphens}{url}
\PassOptionsToPackage{dvipsnames,svgnames*,x11names*}{xcolor}
%
\documentclass[
]{article}
\usepackage{lmodern}
\usepackage{amssymb,amsmath}
\usepackage{ifxetex,ifluatex}
\ifnum 0\ifxetex 1\fi\ifluatex 1\fi=0 % if pdftex
  \usepackage[T1]{fontenc}
  \usepackage[utf8]{inputenc}
  \usepackage{textcomp} % provide euro and other symbols
\else % if luatex or xetex
  \usepackage{unicode-math}
  \defaultfontfeatures{Scale=MatchLowercase}
  \defaultfontfeatures[\rmfamily]{Ligatures=TeX,Scale=1}
  \setmainfont[]{Helvetica}
  \setmonofont[Scale=0.8]{Source Code Pro}
\fi
% Use upquote if available, for straight quotes in verbatim environments
\IfFileExists{upquote.sty}{\usepackage{upquote}}{}
\IfFileExists{microtype.sty}{% use microtype if available
  \usepackage[]{microtype}
  \UseMicrotypeSet[protrusion]{basicmath} % disable protrusion for tt fonts
}{}
\makeatletter
\@ifundefined{KOMAClassName}{% if non-KOMA class
  \IfFileExists{parskip.sty}{%
    \usepackage{parskip}
  }{% else
    \setlength{\parindent}{0pt}
    \setlength{\parskip}{6pt plus 2pt minus 1pt}}
}{% if KOMA class
  \KOMAoptions{parskip=half}}
\makeatother
\usepackage{xcolor}
\IfFileExists{xurl.sty}{\usepackage{xurl}}{} % add URL line breaks if available
\IfFileExists{bookmark.sty}{\usepackage{bookmark}}{\usepackage{hyperref}}
\hypersetup{
  colorlinks=true,
  linkcolor=white,
  filecolor=Maroon,
  citecolor=Blue,
  urlcolor=cyan,
  pdfcreator={LaTeX via pandoc}}
\urlstyle{same} % disable monospaced font for URLs
\setlength{\emergencystretch}{3em} % prevent overfull lines
\providecommand{\tightlist}{%
  \setlength{\itemsep}{0pt}\setlength{\parskip}{0pt}}
\setcounter{secnumdepth}{-\maxdimen} % remove section numbering
\usepackage{unicode-math}

\date{}

\begin{document}

\hypertarget{odes-ii-rungekutta-methods}{%
\section{15. ODEs II: Runge--Kutta
methods}\label{odes-ii-rungekutta-methods}}

\hypertarget{last-time}{%
\subsection{Last time}\label{last-time}}

\begin{itemize}
\item
  Review of ODEs
\item
  Euler method and error
\item
  Trapezium method
\item
  Systems of equations
\end{itemize}

\hypertarget{goals-for-today}{%
\subsection{Goals for today}\label{goals-for-today}}

\begin{itemize}
\item
  Systems of equations
\item
  Higher derivatives
\item
  Runge--Kutta methods
\item
  Adaptivity
\end{itemize}

\hypertarget{systems-of-equations}{%
\subsection{Systems of equations}\label{systems-of-equations}}

\begin{itemize}
\item
  General system of 1st-order ODEs:

  \begin{align*}
    \dot{x_1} &= f_1(t, x_1, \ldots, x_n) \\
    \dot{x_2} &= f_2(t, x_1, \ldots, x_n) \\
    &\vdots  \\
    \dot{x_n} &= f_n(t, x_1, \ldots, x_n) \\
    \end{align*}

  . . .
\item
  Rewrite in vector form:
  \(\dot{\mathbf{x}}(t) = \mathbf{f}(t, \mathbf{x}(t))\)
\item
  \(\mathbf{f} = (f_1, \ldots, f_n)\)
\item
  \(\mathbf{x}(t) = (x_1(t), x_2(t), \ldots, x_n(t))\)
\item
  \(\mathbf{f}\) is a \textbf{vector field}
\end{itemize}

\hypertarget{solving-systems-of-equations}{%
\subsection{Solving systems of
equations}\label{solving-systems-of-equations}}

\begin{itemize}
\item
  \(\mathbf{x}(t) = (x_1(t), x_2(t), \ldots, x_d(t))\) if \(d\)
  variables
\item
  Taylor expand:

  \begin{align*}
    x_i(t_k + h) &= x_i(t_k) + h \, \dot{x_i}(t_k) + \mathcal{O}(h^2) \\
      &= x_i(t_k) + h \, f_i(t_k, x_1, \ldots, x_n) + \mathcal{O}(h^2)
     \end{align*}
\item
  So obtain Euler method

  \[\mathbf{x}_{k+1} = \mathbf{x}_k + h \, \mathbf{f}(\mathbf{x}_k)\]

  . . .
\item
  \emph{Same} method but now with vectors
\item
  \emph{Same} code
\end{itemize}

\hypertarget{higher-derivatives}{%
\subsection{Higher derivatives}\label{higher-derivatives}}

\begin{itemize}
\item
  How treat higher-order equations (higher derivatives)?
\item
  E.g. damped harmonic oscillator

  \[\ddot{x} + b \dot{x} + \omega^2 x = 0\]

  . . .

  ~\\
\item
  There are some special methods for second-order
\end{itemize}

\hypertarget{higher-derivatives-ii}{%
\subsection{Higher derivatives II}\label{higher-derivatives-ii}}

\begin{itemize}
\item
  Usually \textbf{reduce} to system of 1st-order equations:

  . . .
\item
  Introduce new variable \(v := \dot{x}\)
\item
  Then \(\dot{v} = \ddot{x}\)

  . . .

  ~\\
\item
  So get system

  \begin{align*}
        \dot{x} &= v \\
        \dot{v} &= -b v + \omega^2 x
    \end{align*}
\end{itemize}

\hypertarget{reducing-error}{%
\subsection{Reducing error}\label{reducing-error}}

\begin{itemize}
\item
  Euler method is globally \(\mathcal{O}(h)\)
\item
  Trapezium method is getter, \(\mathcal{O}(h^2)\), but is
  \textbf{implicit} so more expensive
\item
  Are there explicit methods of higher order?
\item
  How could we produce one?

  . . .

  ~\\
\item
  Yes: Runge--Kutta and multistep methods
\item
  We will look at Runge--Kutta methods
\end{itemize}

\hypertarget{second-order-taylor-methods}{%
\subsection{Second-order Taylor
methods}\label{second-order-taylor-methods}}

\begin{itemize}
\item
  To get a higher-order method, use a higher-order Taylor expansion for
  a step
\item
  Suppose \(\dot{x}(t) = f(t, x(t))\)
\item
  Expand to higher order:

  \[x(t + h) = x(t) + h \dot{x}(t) + \textstyle \frac{1}{2} h^2 \ddot{x}(t) + \mathcal{O}(h^3)\]
\item
  How can we deal with \(\ddot{x}(t)\)?
\end{itemize}

\hypertarget{second-order-taylor-methods-ii}{%
\subsection{Second-order Taylor methods
II}\label{second-order-taylor-methods-ii}}

\begin{itemize}
\item
  Differentiate the ODE:

  \[\ddot{x}(t) = \textstyle \frac{d}{dt} \left[ f(t, x(t)) \right]\]

  . . .
\item
  Get \(\ddot{x}(t) = f_t + f_x \dot{x}\)
\item
  Where \(f_t := \frac{\partial{f}}{\partial t}(t, x(t))\) and similarly
  for \(f_x\)
\item
  So

  \[x(t + h) \simeq x + h \, f + \textstyle \frac{1}{2} h^2 \left[ f_t + f_x \, f \right]\]
\item
  For clarity write \(f\) instead of \(f(t, x(t))\), etc. and write
  \(x\) for \(x(t)\)
\item
  This is useful \emph{only} if we can calculate the derivatives of
  \(f\)
\end{itemize}

\hypertarget{back-to-the-trapezium-rule}{%
\subsection{Back to the trapezium
rule}\label{back-to-the-trapezium-rule}}

\begin{itemize}
\item
  Let's look at the trapezium rule again:

  \[ x_{n+1} \simeq x_n + \textstyle \frac{h}{2} \left[ f(x_n) + f(x_{n+1}) \right]\]
\item
  Implicitness from \(f(x_{n+1})\) on the right
\item
  Could we approximate this to make it explicit?

  . . .

  ~\\
\item
  Use the Euler method! i.e.~take an \textbf{Euler step}:

  \[x_{n+1} \simeq x_n + h \, f(x_n)\]
\item
  Use this only for term on the right
\end{itemize}

\hypertarget{rungekutta}{%
\subsection{Runge--Kutta}\label{rungekutta}}

\begin{itemize}
\item
  This gives a \textbf{multistage} method.
\item
  Use notation \(k_i\) for different evaluations of \(f\) in sequence:

  \begin{align*}
    k_1 :=& f(t_n, x_n)  \\
    k_2 :=& f(t_n + h, x_n + h \, k_1) \\
    x_{n+1} =& x_n + \textstyle \frac{h}{2} (k_1 + k_2)
    \end{align*}
\item
  \textbf{Modifed Euler method}
\end{itemize}

\hypertarget{order-of-modified-euler}{%
\subsection{Order of modified Euler}\label{order-of-modified-euler}}

\begin{itemize}
\item
  Taylor expand:

  \[k_2 = f + h \, f_t + h \, k_1 \, f_x + \mathcal{O}(h^2)\]
\item
  So \begin{align*}
    x_{n+1} &= x_n + \textstyle \frac{h}{2} (k_1 + k_2) \\
    & \simeq x_n + \textstyle \frac{h}{2} \left[ f + f + h f_t + h k_1 f_x + \mathcal{O}(h^2) \right] \\
    &= x_n + h \, f + \textstyle \frac{1}{2} h^2 (f_t + f_x f) + \mathcal{O}(h^3)
    \end{align*}
\item
  2nd-order Taylor expansion by nested evaluation of \(f\)!
\end{itemize}

\hypertarget{general-rungekutta-methods}{%
\subsection{General Runge--Kutta
methods}\label{general-rungekutta-methods}}

\begin{itemize}
\item
  General idea: Match Taylor expansion
\item
  Add more coefficients in more \textbf{stages}

  \begin{align*}
    k_1 &= f(t_n, x_n) \\
    k_2 &= f(t_n + c_1 \, h, x_n + a_{11} k_1 \\
    k_3 &= f(t_n + c_2 \, h, x_n + a_{21} k_1 + a_{22} k_2 \\
        \vdots \\
    k_s &= f(t_n + c_1 \, h, x_n + a_{s-1,1} k_1 + \cdots + a_{s-1,s-1} k_{s-1} \\
    x_{n+1} &= x_n + b_1 k_1 + \cdots + b_s k_s
    \end{align*}
\item
  Coefficients must satisfy certain constraints
\end{itemize}

\hypertarget{butcher-tableaux}{%
\subsection{Butcher tableaux}\label{butcher-tableaux}}

\begin{itemize}
\tightlist
\item
  Arrange coefficients of method in a \textbf{Butcher tableau}:
\end{itemize}

\[
\begin{array}{c|cccc}
0\\
c_1 & a_{11}\\
c_2 & a_{21} & a_{22} \\
\vdots \\
c_{s-1} & a_{s-1,1} & \cdots & a_{s-1, s-1} \\
\hline
& b_1 & b_2 & b_{s-1} & b_s
\end{array}
\]

\begin{itemize}
\tightlist
\item
  E.g. for modified Euler:
\end{itemize}

\[
\begin{array}{c|cccc}
0\\
1 & 1\\
\hline
& \frac{1}{2} & \frac{1}{2}
\end{array}
\]

\end{document}
